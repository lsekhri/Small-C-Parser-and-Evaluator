\documentclass[10pt]{article}

\usepackage{semantic}
\usepackage{titlesec}

% \title{Project 3}
% \subtitle{SmallC Formal Operational Semantics}

\date{Fall 2019}

\begin{document}
\title{%
  SmallC Formal Operational Semantics}

\maketitle

\newcommand{\config}[2][A]{{#1};{#2}}
\titleformat{\subsection}[runin]{\bfseries \itshape}{\thesubsection.}{3pt}{\space}[.]
\titlespacing*{\subsection}{\parindent}{1ex}{1em}

\section{Introduction}

This document presents a formal semantics of Small C. It has two main
parts. The first part is the \emph{expression semantics}. It
corresponds to the implementation of your \texttt{eval\_expr}
function. The second part is the \emph{statement semantics}. It
corresponds to the implementation of your \texttt{eval\_stmt}
function.

\section{Preliminaries}

\subsection{Environments}

Both parts define judgments that make use of environments $A$. The
rules show two operations on environments:
\begin{itemize}
\item $A(x)$ means to look up the value of $x$ maps to in the
  environment $A$. This operation is undefined if there is no mapping
  for $x$ in $A$.  We write $\cdot$ for the empty environment. Lookup
  on this environment is always undefined; i.e., $\cdot(x)$ is
  undefined for all $x$.
\item The second operation is written $A[ x \mapsto v ]$. It defines a
  new environment that is the same as $A$ but maps $x$ to $v$. It thus
  overrides any prior mapping for $x$ in $A$. This is similar to the
  ``concatenation'' of environments shown in the lecture notes from
  Feb. 28. So, $\cdot[ x \mapsto 1 ]$ is an environment that maps $x$
  to $1$ but is undefined for all other variables. The environment
  $\cdot[ x \mapsto 1][y \mapsto 2]$ is an environment that maps $x$
  to $1$ and $y$ to 2. As such, if
  $A = \cdot[ x \mapsto 1][y \mapsto 2]$ then $A(x) = 1$. That is,
  looking up $x$ in the second example environment produces its
  mapped-to value, 1.
\end{itemize}

\subsection{Syntax}

In this document, we have simplified the presentation of the syntax,
so it may not correspond exactly to the files that your interpreter
will read in. For example, we write $\mathtt{while}\;e\,s$ to
represent the syntax of a while-loop, where $e$ is the guard and $s$
is the body. This corresponds to \texttt{While of expr * stmt} in the
\texttt{types.ml} file. Hopefully the connection between what we show
here at that file is clear enough from context.

\subsection{Error conditions}

The semantics here defines only \emph{correct} evaluations. It says
nothing about what happens when, say, you have a type error. For
example, for the rules below there is no value $v$ for which you can
prove the judgment $\config[\cdot]{1+\mathbf{true}} --> v$. In your actual
implementation, erroneous programs will cause an exception to be
raised, as indicated in the project README.

\section{Expression Semantics}

This part of the formal semantics corresponds to the implementation of
your \texttt{eval\_expr} function. That function has type
\texttt{environment -> expr -> value}. In the semantics, we write the
judgment $\config{e} --> v$, where $A$ represents the environment, $e$
represents the expression, and $v$ represents the final value. This
follows the same format as the lecture notes from February 28.

\subsection{Abstract syntax grammar}

Expressions and values are defined by the following grammar:
\[
\begin{array}{lrcl}
\mathrm{Integers} & n & \mathit{is} & \mathrm{any\ integer}\\
\mathrm{Booleans} & b & \mathtt{::=} & \mathbf{true} \mid \mathbf{false} \\
\mathrm{Values} & v & \mathtt{::=} & n \mid b \\
\mathrm{Expressions} & e & \mathtt{::=} & v \mid\ !e \mid e \oplus e \mid e \odot e \mid e == e \mid
                   e\ != e \\
\end{array}
\]
Here, we write $\oplus$ to represent any operator involving a pair of
integers. This could be addition ($+$), comparison ($\leq$),
multiplication ($\times$), etc. We write $\odot$ to represent any
operator involving a pair of booleans. This could be boolean-and
($\&\&$), boolean-or ($||$), etc. We do not go into specifics here
about what these actually do; they should match your
intuition.

\subsection{Rules}

%\newcommand\BrText[2]{%
%  \par\smallskip
%   \noindent\makebox[\textwidth][r]{$\text{#1}\left\{
%    \begin{minipage}{\textwidth}
%    #2
%    \end{minipage}
%  \right.\nulldelimiterspace=0pt$}\par\smallskip
%}

Here are the axioms.

$$
\begin{array}{c}

\inference[Id]
{A(x) = v}
{\config{x} --> v}
\qquad

\inference[Int]
{}
{\config{n} --> n}
\\ \\

\inference[Bool-True]
{}
{\config{\mathbf{true}} --> \mathbf{true}}

\qquad

\inference[Bool-False]
{}
{\config{\mathbf{false}} --> \mathbf{false}}
\\ \\ \\

\end{array}
$$

Here are the rest of the rules for expressions.

$$
\begin{array}{c}

\inference[Eq-True]
{\config{e_1} --> v \\
\config{e_2} --> v}
{\config{e_1 == e_2} --> \mathbf{true}}

\qquad

\inference[Eq-False]
{\config{e_1} --> v_1 \\
\config{e_2} --> v_2 \\
v_1 \mathrm{~is~different\ than~} v_2}
{\config{e_1 == e_2} --> \mathbf{false}}

\\ \\

\inference[NotEq-True]
{\config{e_1} --> v_1 \\
\config{e_2} --> v_2 \\
v_1 \mathrm{~is~different\ than~} v_2}
{\config{e_1\ != e_2} --> \mathbf{true}}

\qquad

\inference[NotEq-False]
{\config{e_1} --> v \\
\config{e_2} --> v}
{\config{e_1\ != e_2} --> \mathbf{false}}

\\ \\

\inference[BinOp-Int]
{\config{e_1} --> n_1 \\
\config{e_2} --> n_2 \\
v \mathrm{~is~} n_1 \oplus n_2}
{\config{e_1 \oplus e_2} --> v}

\qquad

\inference[BinOp-Bool]
{\config{e_1} --> b_1 \\
\config{e_2} --> b_2 \\
b_3 \mathrm{~is~} b_1 \odot  b_2}
{\config{e_1 \odot e_2} --> b_3}

\\ \\ \\

\inference[Unary-Not]
{\config{e_1} --> b_1 \\
b_2 \mathrm{~is~} \neg b_1}
{\config{! e_1} --> b_2}

\\ \\ \\

\end{array}
$$

\section{Statement Semantics}

This part of the formal semantics corresponds to the implementation of
your \texttt{eval\_stmt} function. That function has type
\texttt{environment -> stmt -> environment}. In the semantics, we write the
judgment $\config{s} --> A'$, where $A$ represents the input environment, $s$
represents the statement to execute, and $A'$ represents the final output
environment. Statements are different from expressions in that they do
not ``return'' anything themselves; instead, their impact occurs by
modifying the environment (by assigning to variables).

\subsection{Abstract syntax grammar}

Statements are defined by the following grammar:
\[
\begin{array}{lrcl}
\mathrm{Statements} & s & \mathtt{::=} & \mathtt{skip} \mid  s
                                         \mathtt{;} s \mid
                                         \mathtt{if}\;e\,s\,s \mid
                                         \mathtt{while}\;e\,s \mid
                                         \mathtt{for}\;x\,e\,e\,s \mid
                                         \mathtt{int}\;x \mid
                                         \mathtt{bool}\;x \mid
                                         x = e \\
\end{array}
\]

In the project itself there is also a statement form for printing; we
leave that out of the formal semantics.

\subsection{Rules}

\subsection*{Variables}

In Small C, you have to declare a variable, with its type, before you
use it. When declared, it will be initialized to a default value. You
cannot declare the same variable twice.

$$
\begin{array}{c}

\inference[Declare-Int]
{A(x) \mathrm{~is~not~defined}}
{\config{\mathtt{int}\;x} --> A[ x \mapsto 0 ]}

\qquad

\inference[Declare-Bool]
{A(x) \mathrm{~is~not~defined}}
{\config{\mathtt{bool}\;x} --> A[ x \mapsto \mathbf{false} ]} \\ \\
\end{array}
$$

Assigning to variables must respect their declared type. In
particular, you cannot assign to a variable you have not declared, and
you cannot write a boolean to a variable declared as an int, or vice versa.

$$
\begin{array}{c}

\inference[Assign-Int]
{A(x) = n\;\mathrm{(for~some~}n)\\
\config{e} --> n_1}
{\config{x=e} --> A[ x \mapsto n_1 ]}
\qquad

\inference[Assign-Bool]
{A(x) = b\;\mathrm{(for~some~}b)\\
\config{e} --> b_1}
{\config{x=e} --> A[ x \mapsto b_1 ]} \\ \\

\end{array}
$$

\subsection*{Control Flow} Here are the rules for the different
control flow constructs.

$$
\begin{array}{c}
\inference[Nop]
{}
{\config{\mathtt{skip}} --> A}
\qquad

\inference[Sequence]
{\config{s_1} --> A_1 \\
\config[A_1]{s_2} --> A_2}
{\config{s_1; s_2} --> A_2}
\\ \\

\inference[If-True]
{\config{e} --> \mathbf{true}\\
\config{s_1} --> A_1}
{\config{\mathtt{if}\; e\, s_1\, s_2} --> A_1}

\qquad

\inference[If-False]
{\config{e} --> \mathbf{false}\\
\config{s_2} --> A_2}
{\config{\mathtt{if}\; e\, s_1\, s_2} --> A_2}

\\ \\ \\

\inference[While-True]
{\config{e} --> \mathbf{true}\\
\config{s} --> A_1\\
\config [A_1]{\mathtt{while}\;e\, s} --> A_2}
{\config{\mathtt{while}\; e\, s} --> A_2}

\qquad

\inference[While-False]
{\config{e} --> \mathbf{false}}
{\config{\mathtt{while}\; e\, s} --> A}

\\ \\ \\

\inference[For-Go]
{\config{e_1} --> n_1\\
 \config{e_2} --> n_2\\
 \config[A \lbrack x \mapsto n_1 \rbrack] {s} --> A_1 \\
 \config [A_1] {x = x + 1} --> A_2 \\
 \config [A_2]{\mathtt{for}\; x\,x\,n_2\,s} --> A' \\
 n_1 \leq n_2}
{\config {\mathtt{for}\; x\,e_1\,e_2\,s} --> A'}

\qquad

\inference[For-Stop]
{\config{e_1} --> n_1\\
 \config{e_2} --> n_2\\
 n_1 > n_2}
{\config {\mathtt{for}\; x\,e_1\,e_2\,s} --> A \lbrack x \mapsto n_1 \rbrack}

\\ \\ \\

\end{array}
$$

\end{document}
